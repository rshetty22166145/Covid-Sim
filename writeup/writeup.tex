\documentclass{article}
\usepackage[utf8]{inputenc}
\usepackage{tabularx}
\usepackage{graphicx}
\usepackage{adjustbox}
\usepackage{changepage}

\title{\textbf{\underline{Simulating the Spread of COVID-19}}}
\author{Aleksey Panas, Anand Singh, Jannate Rahim, Rohit Shetty}
\date{\emph{December 2021}}

\begin{document}

\maketitle

\section{Project Description}

COVID-19 is a disease caused by a virus called SARS-CoV-2. This disease emerged around December 2019 causing millions of deaths all across the world. The virus spreads from an infected person to others through respiratory droplets and aerosols. The practice of social distancing and vaccines are factors that helped reduce the spread of the virus. Much data has been collected to analyze the exponential spread of Covid-19, and assist with slowing it . The data sets are crucial in scientific research as they allow us to test or check future insights and will be quite necessary for future long-term study.
\bigskip

The goal of our project is to present a simplified model that will simulate the spread of an epidemic in a given population. With this model, we will try to simulate some real-world conditions and interpret how these conditions affect the spread of an epidemic(Covid-19) in this case. 
\bigskip

A ‘curve’ in our context is a graphical representation of the onset of illness among cases associated with an outbreak.
The term “Flatten the Curve” means to spread out the rate of infection in order to prevent an overwhelmed health care system and infrastructure. During our research, we will explore many factors that have an effect on the curve. We will also analyze how each of the preventative measures (social distancing, quarantine, vaccinations, etc) and socio-economic factors (population density, access to healthcare, etc) affect this curve. 
\bigskip

The question we want to ask is, \textbf{given the nature of a particular urban center or town, can we accurately simulate the outcome of a Covid outbreak if it were to happen there} and identify causing factors which lead to an undesirable “curve” outcome?

\bigskip
With gradual improvement of the simulation through feedback from real world data and scientific research, we might get a usable tool for predicting the risk an epidemic poses on regions with particular characteristics. This would allow testing policies in advance, and foreseeing the leading causes of the spread of infection.


\section{Relevant Dataset}

A  relevant  dataset  that we used for our  project  comes
from the Government of Canada.  This dataset holds information about various Toronto neighborhoods and their CoVID data. The dataset is available in many formats,  including  CSV  and  JSON,  and  contains  statistics  on  the  population of  each  neighborhood, such as  the  percent  of  elderly  people, density  of  the population, percent young people, and other similar numbers.  Alongside this, the  dataset  shows  CoVID  case  progression,  testing  practices,  and  outbreaks. We used the variables population\_density, percent\_over\_65/85, and neighborhood area, in order to run a simulation with similar parameters as this data. This will then allow us  to see a similar covid case rate and possibly be able to conclude that there was a relationship between the variables.


\bigskip

\textbf{Link to Dataset:} 

https://open.canada.ca/data/en/dataset/2d86f026-10b4-44ac-a68b-80a9dd5dd390/resource/ab558292-2e62-4b71-944b-aa6c19cc5d41?inner\_span=True

\bigskip


\bigskip


\section{Computational Overview}
\textit{To-Do}
\bigskip


\section{Instructions for Obtaining Sets and Running Program}
\textit{To-Do}
\bigskip

\section{Changes Made to Project Plan}
The initial goal of our project was to present a simplified model that would simulate the spread of an epidemic in a given population. We would then try to simulate some real-world conditions and interpret how these conditions would affect the spread of an epidemic(Covid-19) in this case. We did consider reversing the problem approach, that is, instead of making a simulation, we would use data to train a prediction model. We deemed that approach would be a bit too complex. Instead, we stuck with the simulation, keeping in mind that it wouldn’t be flawless at first attempt. The simulation will need to be improved over time using surrounding data.
\bigskip

\section{Discussion}
\textit{To do}
\bigskip

\section{Works Cited}

“The University of Alabama at Birmingham.” \emph{UAB}, http://www.uab.edu/.
\bigskip

\emph{Centers for Disease Control and Prevention}, Centers for Disease Control and
Prevention, http://www.cdc.gov/. 
\bigskip 
 
“What Is Coronavirus?” \emph{Johns Hopkins Medicine}, 
https://www.hopkinsmedicine.org/health/conditions-and-diseases/coronavirus. 
\bigskip

“These Simulations Show How to Flatten the Coronavirus Growth Curve.” \emph{The Washington
Post}, WP Company, 14 Mar. 2020, 
https://www.washingtonpost.com/graphics/
2020/world/corona-simulator/. 
\bigskip

“Toronto Air Pollution and Covid-19 Data by Neighbourhood - Toronto Air Pollution and COVID-19 Data by Neighbourhood.” \emph{Open Government Portal} , https://open.canada.ca/data/en/dataset/2d86f026-10b4-44ac-a68b-80a9dd5dd390/resource/ab558292-2e62-4b71-944b-aa6c19cc5d41?innerspan=True. 
\bigskip

“Pygame Front Page¶.” Pygame Front Page - Pygame v2.1.1 Documentation, https://www.pygame.org/docs/. 
\bigskip

“Multiprocessing - Process-Based Parallelism¶.” Multiprocessing - Process-Based Parallelism - Python 3.10.1 Documentation, https://docs.python.org/3/library/multiprocessing.html. 
\bigskip

“Getting Started.” Getting Started - Guizero, https://lawsie.github.io/guizero/start/. 
\bigskip

“Pandas Documentation¶.” Pandas Documentation - Pandas 1.3.5 Documentation, https://pandas.pydata.org/docs/. 
\bigskip

\end{document}

